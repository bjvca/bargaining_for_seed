%% LyX 2.3.6 created this file.  For more info, see http://www.lyx.org/.
%% Do not edit unless you really know what you are doing.
\documentclass{article}
\usepackage[latin9]{inputenc}

\makeatletter
%%%%%%%%%%%%%%%%%%%%%%%%%%%%%% User specified LaTeX commands.
\usepackage{graphicx}% Required for inserting images

\title{bargaining_for_seed}
\author{bjorn.vancampenhout }
\date{March 2023}







\makeatother

\begin{document}
\author{Bjorn Van Campenhout\thanks{Development Strategy and Governance Division, IFPRI, Belgium - corresponding
author: bjorn.vancampenhout@cgiar.org}, Olayinka Aremu\thanks{Department of Environmental Systems Science, ETH Zurich,Switzerland}}
\title{Bargaining over seed}
\maketitle

\section{Introduction}

In many societies, it is customary to bargain over prices

\section{Bargaining Experiment}

Farmers are offered the opportunity to buy a bag of seed from an enumerator
in a way that is as close as possible as how this happens in a real
life setting where bargaining is the norm. The enumerator follows
a standard script. An initial ask price is randomly drawn, ranging
from 12,000 to 9,000, and this price is then presented to the farmer
as the price of the bag of seed. The enumerator then explains what
kind of seed it is and what the advantages are. The farmer is then
encourage to name his/her first price. A computer algorithm then determines
a counter-offer that the enumerator asks, reemphasizing why the first
price was too low. The farmer can then make a second bid. Bargaining
continues until the price difference between the bid and ask price
falls below a certain threshold (eg 1000) in which case the computer
instructs the enumerator to indicate that this is his/her last price.
The farmer can then decide to buy. If the farmer does not want to
buy we will still sell the seed at the last bid made by the farmer.
To make the bargaining also incentive compatible for the enumerators,
we tell them in advance that the money that is collected from farmers
during this first stage will be divided and distributed equally among
all the enumerators.

A popular alternative way to measure willingness to pay is a Becker-DeGroot-Marschak
(BDM) auction. In it simplest version, the subject formulates a bid
and this bid is compared to a price determined by a random number
generator. If the subject's bid is greater than the price, they pay
the price and receives the item being auctioned. If the subject's
bid is lower than the price, they pay nothing and receive nothing.
However, after testing in the field, we found that too many farmers
had problems comprehending the procedure, struggling especially with
the fact that they could not bargain over the price.
\end{document}
